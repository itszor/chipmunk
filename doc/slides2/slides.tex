\documentclass[fancybox]{seminar}

\usepackage{semcolor}
\input{seminar.bug}
\usepackage{pstcol}
\usepackage{pst-grad}
\usepackage{slidesec}
\usepackage{fancybox}


\newenvironment{code}
  {\begin{list}{}{
    \setlength{\rightmargin}{\leftmargin}
    \raggedright
    \setlength{\itemsep}{0pt}
    \setlength{\parsep}{0pt}
    \ttfamily}%
   \item[]}
  {\end{list}}

\newcommand{\heading}[1]{%
  \begin{center}
    \Ovalbox{\large\bf #1}
  \end{center}
  \vspace{1ex minus 1ex}}

\definecolor{Gold}{rgb}{1.,0.84,0.}
\slideframe{scplain}

\begin{document}

\begin{slide}

\heading{What I'm doing}

\begin{itemize}
\item Research into {\em dynamic optimisation}
\item In particular, support features in an instruction-set
\end{itemize}

\end{slide}


\begin{slide}

\heading{What is dynamic optimisation?}

\begin{itemize}
\item Try to improve code as it is running
\item Use continuously-gathered profile data
\item Normally perform optimisations in the absence of source code
\item No need for additional compilation stage
\end{itemize}

\end{slide}


\begin{slide}

\heading{New/improved work in field}

\begin{itemize}
\item HP Dynamo
\item MS Research Mojo
\item Wiggins/Redstone
\item Deco
\item (P4 trace cache)
\end{itemize}

\end{slide}


\begin{slide}

\heading{HP Dynamo}

\begin{itemize}
\item Ported to new architectures (SH4, ARM, x86) and operating systems (Linux, Windows CE)
\item Aiming for embedded applications
\item Pre-emptive cache flushing
\item Low footprint
\item Not much info available on newer versions
\end{itemize}

\end{slide}


\begin{slide}

\heading{MS Research Mojo}

\begin{itemize}
\item Aims to optimise large, multi-threaded Windows applications
\item Initial results disappointing
\item Attempted use of x86 arch a problem?
\end{itemize}

\end{slide}


\begin{slide}

\heading{Wiggins/Redstone}

\begin{itemize}
\item Gathers data values as well as program flow
\item Specialises binary code
\item Generates traces long enough to insert prefetch instructions
\item Seemed to obtain good results with POV-ray renderer
\end{itemize}

\end{slide}


\begin{slide}

\heading{Common themes}

\begin{itemize}
\item ``Hot'' trace flattening
\item Copy propagation
\item Partial dead code removal
\item Load-store alias detection
\item Load-verify
\item Prefetch insertion
\end{itemize}

\end{slide}


\begin{slide}

\heading{Assorted architecture features}

\begin{itemize}
\item Based around superblocks rather than flat code or basic blocks (as previously)
\item Memory access tags
\item Basic type information
\item User-space memory protection
\item Continuous profiling in hardware
\item Old features (register expiration, etc.)
\end{itemize}

\end{slide}


\begin{slide}

\heading{Superblocks}

\begin{itemize}
\item Code arranged as sequences with:
  \begin{itemize}
  \item One entry point
  \item Multiple exit points
  \item No internal branches
  \end{itemize}
\item Allow contiguous ``hot'' regions to be formed
\item All code pointers dereferenced
\item Previous arrangement broke instruction caching
\end{itemize}

\end{slide}


\begin{slide}

\heading{Memory access tags}

\begin{itemize}
\item Indication of how program can be safely altered
\item Might occupy 2 bits per access instruction
\item Can indicate, eg, if a store must be performed to preserve semantics of program
\item Probably useful for:
  \begin{itemize}
  \item Register reallocation
  \item Safely removing stack frame code
  \item Alias detection
  \end{itemize}
\end{itemize}

\end{slide}


\begin{slide}

\heading{Type information}

\begin{itemize}
\item Store at suitable points in program
\item Might be fairly crude, eg:
  \begin{itemize}
  \item Scalar
  \item Pointer to area of particular size
  \item Something ``more complicated''
  \end{itemize}
\item Need to hijack {\tt malloc}, etc.?
\end{itemize}

\end{slide}


\begin{slide}

\heading{Fine-grained memory protection}

\begin{itemize}
\item Small number of write-protected areas
\item User-space fault handler
\item Use to optimistically special-case loops with possible aliases
  \begin{itemize}
  \item Allows aggressive reordering of loads and stores
  \item Only done if a loop is time-consuming
  \end{itemize}
\item Combine with memory access tags \& type information
\item General applicability?
\end{itemize}

\end{slide}


\begin{slide}

\heading{Example}

\begin{code}
void vsum(float a[], float b[], float c[], \\
\ \ \ \ \ \ \ \ \ \ int size)\\
\{\\
\ \ int i;\\
\ \ for (i=0; i < size; i++)\\
\ \ \ \ a[i] = b[i] + c[i];\\
\}\\

float vec[100];\\
vec[0] = vec[1] = 1.0;\\
vsum(\&vec[2], \&vec[0], \&vec[1], 97);
\end{code}

\end{slide}


\begin{slide}

\heading{Continuous profiling}

\begin{itemize}
\item Most time-consuming part of current dynamic optimisation systems
\item Need to look for ways to do accurately and efficiently
\item The right hardware-based approach could probably be pipelined
\item ``Level of detail'' profiling?
\end{itemize}

\end{slide}


\begin{slide}

\heading{Speeding up interpreters}

\begin{itemize}
\item Lots of programs based around a bytecode-interpreting core
\item Tend to run slowly (Python, non-JIT Java, Perl...)
\item Look for a way of exploiting pattern
  \begin{itemize}
  \item Big {\tt switch} statement
  \item Each case may contain function calls, etc. (Python)
  \end{itemize}
\item Will Wiggins/Redstone approach do by itself?
\item Lexing, etc. hopefully similar enough to also benefit
\end{itemize}

\end{slide}


\end{document}
